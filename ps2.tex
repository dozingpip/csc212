\documentclass[11pt]{article}

\usepackage[margin=.85in]{geometry}
\usepackage{amsmath, amsfonts, amssymb}
\usepackage{enumitem}
\usepackage{listings}
\usepackage{color}
\usepackage{graphicx}

\graphicspath{ {images/} }
\setlength{\parindent}{0pt}
\setlength{\fboxsep}{10pt}

\definecolor{mygreen}{rgb}{0,0.6,0}
\definecolor{mygray}{rgb}{0.5,0.5,0.5}
\definecolor{mymauve}{rgb}{0.58,0,0.82}

\lstset{ %
    backgroundcolor=\color{white},   % choose the background color
    basicstyle=\footnotesize,        % size of fonts used for the code
    breaklines=true,                 % automatic line breaking only at whitespace
    commentstyle=\color{mygreen},    % comment style
    keywordstyle=\color{blue},       % keyword style
    stringstyle=\color{mymauve},     % string literal style
}
% --------------------------------------------------------------------------
% --------------------------------------------------------------------------
% --------------------------------------------------------------------------
\begin{document}

% --------------------------------------------------------------------------
\begin{center}
{\Large\bf CSC 212: Data Structures and Abstractions}\\
\medskip
{\Large\bf University of Rhode Island}\\
\bigskip
{\Large\bf Problem Set 02 (Fall 2016)}
\end{center}

% --------------------------------------------------------------------------
\bigskip
\fbox{\parbox{6.5in}{
    \vspace{12pt}
    \textbf{\large Stephanie Donnelly}
    \vspace{12pt}
}}
\bigskip

% --------------------------------------------------------------------------
% --------------------------------------------------------------------------
% --------------------------------------------------------------------------
\begin{enumerate}[leftmargin=*]

    % problem 1
    % --------------------------------------------------------------------------
    \item

    \begin{enumerate}
	%a
        \item
	\begin{align*}
           O(nlogn)\\
    	\end{align*}

	%b
        \item
	\begin{align*}
            O(n)\\
    	\end{align*}

	%c
        \item
	\begin{align*}
            O(n2^n)\\
    	\end{align*}

	%d
        \item
	\begin{align*}
            O(n2^n)\\
    	\end{align*}

	%e
        \item
	\begin{align*}
            O(logn)\\
    	\end{align*}

	%f
        \item
	\begin{align*}
            O(n^2)\\
    	\end{align*}
	\end{enumerate}
    % problem 2
    % --------------------------------------------------------------------------
    \item {\it }
	\begin{align*}
        O(logn)\\
    	\end{align*}

    % problem 3
    % --------------------------------------------------------------------------
    \item {\it }

	\includegraphics{prob3.jpg}

    % problem 4
    % --------------------------------------------------------------------------
    \item {\it }

    \verb|listings| environment:
    \begin{lstlisting}[language=C++]
        long int tribonacci(int n) {
            long int current = 0;
            long int prev= 0;
            long int prev_prev = 0;
            for(int i = 0; i<n; i++){
                current = num;
                num = current + prev + prev_prev;
                prev  = num;
                prev_prev = num - prev;
            }
        }
    \end{lstlisting}

    % problem 5
    % --------------------------------------------------------------------------
    \item {\it }

    \verb|listings| environment:
    \begin{lstlisting}[language=C++]
        void reverse(char *str, int n) {
            for(int i = 0; i<n/2; i++){
                char temp = *str;
                *str = *(str-i)
                *(str-i) = temp;
            }
        }
    \end{lstlisting}

    % problem 6
    % --------------------------------------------------------------------------
    \item {\it }

    \verb|listings| environment:
    \begin{lstlisting}[language=C++]
                void reverse(char *str, int n) {
            for(int i = 0; i<n/2; i++){
                char temp = *str;
                *str = *(str-i)
                *(str-i) = temp;
            }
        }

        int palindrome(char *str, int n) {
            string s = " ";
            while(*str != '\0'){
                s +=*str;
                str++;
            }
            int mid = strlen(s)/2;
            if(strlen(s)%2 ==1){
                mid++;
            }
            string half2 = reverse(s.substr(mid) //get from midpoint on and reverse that
            string half1 = s.substr(0, mid);
            if(half1 == half2){
                return 1;
            }else{
                return 0;
            }
        }
    \end{lstlisting}

    % problem 7
    % --------------------------------------------------------------------------
    \item {\it }

    \verb|listings| environment:
    \begin{lstlisting}[language=C++]
        void sort(int *array, int n) {
            for(int i = 0; i < n; i++){
                if(array[i]>array[i+1]{
					int temp = array[i];
					array[i+1] = array[i];
					array[i] = temp;
				}
            }
        }
    \end{lstlisting}

    % problem 8
    % --------------------------------------------------------------------------
    \item {\it }
    \begin{enumerate}
        \item
            %part a
            correct

        \item
            %part b
            compiler error, can't convert int to int*

        \item
            %part c
            compiler error: don't convert between double* and int*

        \item
            %part d
            correct

        \item
            %part e
            correct

        \item
            %part f
            correct

        \item
            %part g
            compiler error, can't convert int** to int*

        \item
            %part h
            correct

        \item
            %part i
            logical error: n will end up being the address of p2 (a pointer, which is probably useless)

        \item
            %part j
            compiler error, can't convert between double* and int*
    \end{enumerate}

    % problem 9
    % --------------------------------------------------------------------------
    \item {\it }
    0 2 4 6 8 10 12 14 16 18

\end{enumerate}
% --------------------------------------------------------------------------
% --------------------------------------------------------------------------
% --------------------------------------------------------------------------
\end{document}
